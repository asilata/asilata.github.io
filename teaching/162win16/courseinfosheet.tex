\documentclass[9pt]{extarticle}

\usepackage[margin=0.7in]{geometry}
\usepackage{ulem}
\usepackage{hyperref, xcolor}
\definecolor{my-linkcolor}{rgb}{0.75,0,0}
\definecolor{my-citecolor}{rgb}{0,0.5,0}
\definecolor{my-urlcolor}{rgb}{0,0,0.75}
\hypersetup{
    colorlinks, 
    linkcolor={my-linkcolor},
    citecolor={my-citecolor}, 
    urlcolor={my-urlcolor}
}
\urlstyle{same}

\pagestyle{empty}
\begin{document}

\begin{center}{\LARGE MATH 16200 Section 30: Honors Calculus II}\end{center}
\vspace{2em}

\section*{Basic Course Information}
\begin{tabular}{l l}
  Meeting Time:& TR 10:30-11:50\\
  Meeting Location:& Ryerson Physical Laboratory, Room 358\\
  Textbook:&(none)\\
  Website:& \url{http://math.uchicago.edu/~asilata/teaching/162win16/}\\
  Clinic Time:& TBA\\
  Clinic Location:&TBA
\end{tabular}

\section*{Instructors}
\begin{tabular}{l l l}
  Name:&Asilata Bapat& Matthew Creek\\
  Office:&Eckhart Hall, Room 2A&Eckhart Hall, Room 306\\
  Email:&\url{asilata@math.uchicago.edu}& \url{mcreek@math.uchicago.edu}\\
  Phone:&N/A& (773) 702-7346\\
  Office Hours:&TBA & TBA
\end{tabular}

\section*{College Fellow}
\begin{tabular}{l l}
  Name:&Anthony Wang\\
  Office:&``Woodlawn Basement'' (5720 S Woodlawn Ave), Room 5E\\
  Email:&\url{ayw@uchicago.edu}\\
  Phone:&N/A\\
  Office Hours:&TBA
\end{tabular}

\section*{Course Description}
This is the second of three parts of the University of Chicago's honors sequence in one-variable calculus.
The class will be run in the Inquiry-Based Learning (IBL) format. 
More information about this can be found \href{http://math.uchicago.edu/~asilata/teaching/162win16/ibl.pdf}{here}.
The role of the instructors will be to facilitate and moderate discussions, as well as to ask leading questions as needed to advance discussions.

Taking off from the previous quarter, we intend to discuss compactness, continuous functions, field axioms and the construction of real numbers, and fundamentals of calculus on the real numbers.

You will be asked to submit written work, including problem sets and journals.
All written work must be submitted using the \LaTeX\ program.
Students should feel free to work together on all written work, provided that each student independently writes up his or her own solutions for submission.
Late work will not be accepted under any circumstances.

Our college fellow will run a weekly problem-solving clinic for this course.
No new material will be presented in this clinic.
Instead, this clinic is meant to be a forum for students to ask questions regarding the material covered during normal class hours.
Additionally, each instructor and the college fellow will hold two office hours per week.
We are always happy to work with students who are putting forth an effort, so please take full advantage of these opportunities for assistance.

\section*{Grading}
Your grade in this course will be determined by the following criteria in equal measure:
\begin{enumerate}
\item Classroom participation, including presentations as well as active listening (comments and questions when another student is presenting).
\item Submitted written work (including journals and homework problems)
\item Final examination (Thursday 17 March + individual oral examinations)
\end{enumerate}

\section*{Departmental boilerplate}
\uline{It is the policy of the Department of Mathematics that the following rules apply to final exams in all undergraduate mathematics courses:}
\begin{enumerate}
\item \uline{The final exam must occur at the time and place designated on the College Final Exam Schedule}. 
In particular, \uline{no} final exams may be given during the tenth week of the quarter, except in the case of graduating seniors.
\item Instructors are not permitted to excuse students from the scheduled time of the final exam except in cases of an Incomplete.
\end{enumerate}
\end{document}
